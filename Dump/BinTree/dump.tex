\documentclass[a4paper,12pt]{article}

\usepackage[utf8]{inputenc}
\usepackage[russian]{babel}
\usepackage{amsmath}
\usepackage{breqn}
\usepackage{graphicx}
\graphicspath{ {Dump/BinTree/Graphic/} }
\usepackage{amssymb}
\usepackage{caption}
\begin{document}
\begin{dmath}
\text{Просто как швейцарское полено: } \frac {dx}{d}( \sin  x  +  0.1  \cdot  \sin  10  \cdot  x )= 0
\end{dmath}
\begin{dmath}
\text{Перед взятием производной: } \frac {dx}{d}( \sin  x  +  0.1  \cdot  \sin  10  \cdot  x )= 0
\end{dmath}
\begin{dmath}
\text{: } \frac {dx}{d}( \sin  x  +  0.1  \cdot  \sin  10  \cdot  x )= 0
\end{dmath}
\begin{dmath}
\text{По всем математическим утверждениям: } \frac {dx}{d}( 0.1  \cdot  \sin  10  \cdot  x )= 0
\end{dmath}
\begin{dmath}
\text{И тд. и тп.: } \frac {dx}{d}( \sin  10  \cdot  x )= 0
\end{dmath}
\begin{dmath}
\text{Нет никаких проблем в том, что: } \frac {dx}{d}( 10  \cdot  x )= 0
\end{dmath}
\begin{dmath}
\text{И тд. и тп.: } \frac {dx}{d}( x )= 0
\end{dmath}
\begin{dmath}
\text{Этому вас должны были научить в садике: } \frac {dx}{d}( 10 )= 0
\end{dmath}
\begin{dmath}
\text{: } \frac {dx}{d}( 0  \cdot  x  +  10  \cdot  1 )= 0
\end{dmath}
\begin{dmath}
\text{Нет никаких проблем в том, что: } \frac {dx}{d}( 0  \cdot  x  +  10  \cdot  1  \cdot  \cos  10  \cdot  x )= 0
\end{dmath}
\begin{dmath}
\text{По всем математическим утверждениям: } \frac {dx}{d}( 0.1 )= 0
\end{dmath}
\begin{dmath}
\text{Нет никаких проблем в том, что: } \frac {dx}{d}( 0  \cdot  \sin  10  \cdot  x  +  0.1  \cdot  0  \cdot  x  +  10  \cdot  1  \cdot  \cos  10  \cdot  x )= 0
\end{dmath}
\begin{dmath}
\text{: } \frac {dx}{d}( \sin  x )= 0
\end{dmath}
\begin{dmath}
\text{Этому вас должны были научить в садике: } \frac {dx}{d}( x )= 0
\end{dmath}
\begin{dmath}
\text{И тд. и тп.: } \frac {dx}{d}( 1  \cdot  \cos  x )= 0
\end{dmath}
\begin{dmath}
\text{Этому вас должны были научить в садике: } \frac {dx}{d}( 1  \cdot  \cos  x  +  0  \cdot  \sin  10  \cdot  x  +  0.1  \cdot  0  \cdot  x  +  10  \cdot  1  \cdot  \cos  10  \cdot  x )= 0
\end{dmath}
\begin{dmath}
\text{После взятием производной: } \frac {dx}{d}( 1  \cdot  \cos  x  +  0  \cdot  \sin  10  \cdot  x  +  0.1  \cdot  0  \cdot  x  +  10  \cdot  1  \cdot  \cos  10  \cdot  x )= 0
\end{dmath}
\begin{dmath}
\text{Перед созданием ряда Тейлора: } \frac {dx}{d}( 1  \cdot  \cos  x  +  0  \cdot  \sin  10  \cdot  x  +  0.1  \cdot  0  \cdot  x  +  10  \cdot  1  \cdot  \cos  10  \cdot  x )= 0
\end{dmath}
\begin{dmath}
\text{И тд. и тп.: } \frac {dx}{d}( \cos  x  +  0.1  \cdot  10  \cdot  \cos  10  \cdot  x )= 0
\end{dmath}
\begin{dmath}
\text{: } \frac {dx}{d}( 0.1  \cdot  10  \cdot  \cos  10  \cdot  x )= 0
\end{dmath}
\begin{dmath}
\text{Нет никаких проблем в том, что: } \frac {dx}{d}( 10  \cdot  \cos  10  \cdot  x )= 0
\end{dmath}
\begin{dmath}
\text{Просто как швейцарское полено: } \frac {dx}{d}( \cos  10  \cdot  x )= 0
\end{dmath}
\begin{dmath}
\text{И тд. и тп.: } \frac {dx}{d}( 10  \cdot  x )= 0
\end{dmath}
\begin{dmath}
\text{Этому вас должны были научить в садике: } \frac {dx}{d}( x )= 0
\end{dmath}
\begin{dmath}
\text{Нет никаких проблем в том, что: } \frac {dx}{d}( 10 )= 0
\end{dmath}
\begin{dmath}
\text{Нет никаких проблем в том, что: } \frac {dx}{d}( 0  \cdot  x  +  10  \cdot  1 )= 0
\end{dmath}
\begin{dmath}
\text{Просто как швейцарское полено: } \frac {dx}{d}( 0  \cdot  x  +  10  \cdot  1  \cdot  \sin  10  \cdot  x  \cdot  -1 )= 0
\end{dmath}
\begin{dmath}
\text{По всем математическим утверждениям: } \frac {dx}{d}( 10 )= 0
\end{dmath}
\begin{dmath}
\text{Просто как швейцарское полено: } \frac {dx}{d}( 0  \cdot  \cos  10  \cdot  x  +  10  \cdot  0  \cdot  x  +  10  \cdot  1  \cdot  \sin  10  \cdot  x  \cdot  -1 )= 0
\end{dmath}
\begin{dmath}
\text{Этому вас должны были научить в садике: } \frac {dx}{d}( 0.1 )= 0
\end{dmath}
\begin{dmath}
\text{Нет никаких проблем в том, что: } \frac {dx}{d}( 0  \cdot  10  \cdot  \cos  10  \cdot  x  +  0.1  \cdot  0  \cdot  \cos  10  \cdot  x  +  10  \cdot  0  \cdot  x  +  10  \cdot  1  \cdot  \sin  10  \cdot  x  \cdot  -1 )= 0
\end{dmath}
\begin{dmath}
\text{Просто как швейцарское полено: } \frac {dx}{d}( \cos  x )= 0
\end{dmath}
\begin{dmath}
\text{Этому вас должны были научить в садике: } \frac {dx}{d}( x )= 0
\end{dmath}
\begin{dmath}
\text{И тд. и тп.: } \frac {dx}{d}( 1  \cdot  \sin  x  \cdot  -1 )= 0
\end{dmath}
\begin{dmath}
\text{Нет никаких проблем в том, что: } \frac {dx}{d}( 1  \cdot  \sin  x  \cdot  -1  +  0  \cdot  10  \cdot  \cos  10  \cdot  x  +  0.1  \cdot  0  \cdot  \cos  10  \cdot  x  +  10  \cdot  0  \cdot  x  +  10  \cdot  1  \cdot  \sin  10  \cdot  x  \cdot  -1 )= 0
\end{dmath}
\begin{dmath}
\text{Этому вас должны были научить в садике: } \frac {dx}{d}( \sin  x  \cdot  -1  +  0.1  \cdot  10  \cdot  10  \cdot  \sin  10  \cdot  x  \cdot  -1 )= 0
\end{dmath}
\begin{dmath}
\text{: } \frac {dx}{d}( 0.1  \cdot  10  \cdot  10  \cdot  \sin  10  \cdot  x  \cdot  -1 )= 0
\end{dmath}
\begin{dmath}
\text{: } \frac {dx}{d}( 10  \cdot  10  \cdot  \sin  10  \cdot  x  \cdot  -1 )= 0
\end{dmath}
\begin{dmath}
\text{Просто как швейцарское полено: } \frac {dx}{d}( 10  \cdot  \sin  10  \cdot  x  \cdot  -1 )= 0
\end{dmath}
\begin{dmath}
\text{И тд. и тп.: } \frac {dx}{d}( \sin  10  \cdot  x  \cdot  -1 )= 0
\end{dmath}
\begin{dmath}
\text{Этому вас должны были научить в садике: } \frac {dx}{d}( -1 )= 0
\end{dmath}
\begin{dmath}
\text{Этому вас должны были научить в садике: } \frac {dx}{d}( \sin  10  \cdot  x )= 0
\end{dmath}
\begin{dmath}
\text{: } \frac {dx}{d}( 10  \cdot  x )= 0
\end{dmath}
\begin{dmath}
\text{Просто как швейцарское полено: } \frac {dx}{d}( x )= 0
\end{dmath}
\begin{dmath}
\text{Просто как швейцарское полено: } \frac {dx}{d}( 10 )= 0
\end{dmath}
\begin{dmath}
\text{И тд. и тп.: } \frac {dx}{d}( 0  \cdot  x  +  10  \cdot  1 )= 0
\end{dmath}
\begin{dmath}
\text{: } \frac {dx}{d}( 0  \cdot  x  +  10  \cdot  1  \cdot  \cos  10  \cdot  x )= 0
\end{dmath}
\begin{dmath}
\text{Этому вас должны были научить в садике: } \frac {dx}{d}( 0  \cdot  x  +  10  \cdot  1  \cdot  \cos  10  \cdot  x  \cdot  -1  +  \sin  10  \cdot  x  \cdot  0 )= 0
\end{dmath}
\begin{dmath}
\text{По всем математическим утверждениям: } \frac {dx}{d}( 10 )= 0
\end{dmath}
\begin{dmath}
\text{И тд. и тп.: } \frac {dx}{d}( 0  \cdot  \sin  10  \cdot  x  \cdot  -1  +  10  \cdot  0  \cdot  x  +  10  \cdot  1  \cdot  \cos  10  \cdot  x  \cdot  -1  +  \sin  10  \cdot  x  \cdot  0 )= 0
\end{dmath}
\begin{dmath}
\text{По всем математическим утверждениям: } \frac {dx}{d}( 10 )= 0
\end{dmath}
\begin{dmath}
\text{По всем математическим утверждениям: } \frac {dx}{d}( 0  \cdot  10  \cdot  \sin  10  \cdot  x  \cdot  -1  +  10  \cdot  0  \cdot  \sin  10  \cdot  x  \cdot  -1  +  10  \cdot  0  \cdot  x  +  10  \cdot  1  \cdot  \cos  10  \cdot  x  \cdot  -1  +  \sin  10  \cdot  x  \cdot  0 )= 0
\end{dmath}
\begin{dmath}
\text{Этому вас должны были научить в садике: } \frac {dx}{d}( 0.1 )= 0
\end{dmath}
\begin{dmath}
\text{Нет никаких проблем в том, что: } \frac {dx}{d}( 0  \cdot  10  \cdot  10  \cdot  \sin  10  \cdot  x  \cdot  -1  +  0.1  \cdot  0  \cdot  10  \cdot  \sin  10  \cdot  x  \cdot  -1  +  10  \cdot  0  \cdot  \sin  10  \cdot  x  \cdot  -1  +  10  \cdot  0  \cdot  x  +  10  \cdot  1  \cdot  \cos  10  \cdot  x  \cdot  -1  +  \sin  10  \cdot  x  \cdot  0 )= 0
\end{dmath}
\begin{dmath}
\text{И тд. и тп.: } \frac {dx}{d}( \sin  x  \cdot  -1 )= 0
\end{dmath}
\begin{dmath}
\text{Этому вас должны были научить в садике: } \frac {dx}{d}( -1 )= 0
\end{dmath}
\begin{dmath}
\text{По всем математическим утверждениям: } \frac {dx}{d}( \sin  x )= 0
\end{dmath}
\begin{dmath}
\text{По всем математическим утверждениям: } \frac {dx}{d}( x )= 0
\end{dmath}
\begin{dmath}
\text{И тд. и тп.: } \frac {dx}{d}( 1  \cdot  \cos  x )= 0
\end{dmath}
\begin{dmath}
\text{И тд. и тп.: } \frac {dx}{d}( 1  \cdot  \cos  x  \cdot  -1  +  \sin  x  \cdot  0 )= 0
\end{dmath}
\begin{dmath}
\text{Просто как швейцарское полено: } \frac {dx}{d}( 1  \cdot  \cos  x  \cdot  -1  +  \sin  x  \cdot  0  +  0  \cdot  10  \cdot  10  \cdot  \sin  10  \cdot  x  \cdot  -1  +  0.1  \cdot  0  \cdot  10  \cdot  \sin  10  \cdot  x  \cdot  -1  +  10  \cdot  0  \cdot  \sin  10  \cdot  x  \cdot  -1  +  10  \cdot  0  \cdot  x  +  10  \cdot  1  \cdot  \cos  10  \cdot  x  \cdot  -1  +  \sin  10  \cdot  x  \cdot  0 )= 0
\end{dmath}
\begin{dmath}
\text{: } \frac {dx}{d}( \cos  x  \cdot  -1  +  0.1  \cdot  10  \cdot  10  \cdot  10  \cdot  \cos  10  \cdot  x  \cdot  -1 )= 0
\end{dmath}
\begin{dmath}
\text{По всем математическим утверждениям: } \frac {dx}{d}( 0.1  \cdot  10  \cdot  10  \cdot  10  \cdot  \cos  10  \cdot  x  \cdot  -1 )= 0
\end{dmath}
\begin{dmath}
\text{Нет никаких проблем в том, что: } \frac {dx}{d}( 10  \cdot  10  \cdot  10  \cdot  \cos  10  \cdot  x  \cdot  -1 )= 0
\end{dmath}
\begin{dmath}
\text{И тд. и тп.: } \frac {dx}{d}( 10  \cdot  10  \cdot  \cos  10  \cdot  x  \cdot  -1 )= 0
\end{dmath}
\begin{dmath}
\text{Этому вас должны были научить в садике: } \frac {dx}{d}( 10  \cdot  \cos  10  \cdot  x  \cdot  -1 )= 0
\end{dmath}
\begin{dmath}
\text{Просто как швейцарское полено: } \frac {dx}{d}( -1 )= 0
\end{dmath}
\begin{dmath}
\text{Просто как швейцарское полено: } \frac {dx}{d}( 10  \cdot  \cos  10  \cdot  x )= 0
\end{dmath}
\begin{dmath}
\text{По всем математическим утверждениям: } \frac {dx}{d}( \cos  10  \cdot  x )= 0
\end{dmath}
\begin{dmath}
\text{По всем математическим утверждениям: } \frac {dx}{d}( 10  \cdot  x )= 0
\end{dmath}
\begin{dmath}
\text{Этому вас должны были научить в садике: } \frac {dx}{d}( x )= 0
\end{dmath}
\begin{dmath}
\text{Просто как швейцарское полено: } \frac {dx}{d}( 10 )= 0
\end{dmath}
\begin{dmath}
\text{И тд. и тп.: } \frac {dx}{d}( 0  \cdot  x  +  10  \cdot  1 )= 0
\end{dmath}
\begin{dmath}
\text{Этому вас должны были научить в садике: } \frac {dx}{d}( 0  \cdot  x  +  10  \cdot  1  \cdot  \sin  10  \cdot  x  \cdot  -1 )= 0
\end{dmath}
\begin{dmath}
\text{По всем математическим утверждениям: } \frac {dx}{d}( 10 )= 0
\end{dmath}
\begin{dmath}
\text{По всем математическим утверждениям: } \frac {dx}{d}( 0  \cdot  \cos  10  \cdot  x  +  10  \cdot  0  \cdot  x  +  10  \cdot  1  \cdot  \sin  10  \cdot  x  \cdot  -1 )= 0
\end{dmath}
\begin{dmath}
\text{Нет никаких проблем в том, что: } \frac {dx}{d}( 0  \cdot  \cos  10  \cdot  x  +  10  \cdot  0  \cdot  x  +  10  \cdot  1  \cdot  \sin  10  \cdot  x  \cdot  -1  \cdot  -1  +  10  \cdot  \cos  10  \cdot  x  \cdot  0 )= 0
\end{dmath}
\begin{dmath}
\text{По всем математическим утверждениям: } \frac {dx}{d}( 10 )= 0
\end{dmath}
\begin{dmath}
\text{И тд. и тп.: } \frac {dx}{d}( 0  \cdot  10  \cdot  \cos  10  \cdot  x  \cdot  -1  +  10  \cdot  0  \cdot  \cos  10  \cdot  x  +  10  \cdot  0  \cdot  x  +  10  \cdot  1  \cdot  \sin  10  \cdot  x  \cdot  -1  \cdot  -1  +  10  \cdot  \cos  10  \cdot  x  \cdot  0 )= 0
\end{dmath}
\begin{dmath}
\text{Этому вас должны были научить в садике: } \frac {dx}{d}( 10 )= 0
\end{dmath}
\begin{dmath}
\text{Нет никаких проблем в том, что: } \frac {dx}{d}( 0  \cdot  10  \cdot  10  \cdot  \cos  10  \cdot  x  \cdot  -1  +  10  \cdot  0  \cdot  10  \cdot  \cos  10  \cdot  x  \cdot  -1  +  10  \cdot  0  \cdot  \cos  10  \cdot  x  +  10  \cdot  0  \cdot  x  +  10  \cdot  1  \cdot  \sin  10  \cdot  x  \cdot  -1  \cdot  -1  +  10  \cdot  \cos  10  \cdot  x  \cdot  0 )= 0
\end{dmath}
\begin{dmath}
\text{: } \frac {dx}{d}( 0.1 )= 0
\end{dmath}
\begin{dmath}
\text{Просто как швейцарское полено: } \frac {dx}{d}( 0  \cdot  10  \cdot  10  \cdot  10  \cdot  \cos  10  \cdot  x  \cdot  -1  +  0.1  \cdot  0  \cdot  10  \cdot  10  \cdot  \cos  10  \cdot  x  \cdot  -1  +  10  \cdot  0  \cdot  10  \cdot  \cos  10  \cdot  x  \cdot  -1  +  10  \cdot  0  \cdot  \cos  10  \cdot  x  +  10  \cdot  0  \cdot  x  +  10  \cdot  1  \cdot  \sin  10  \cdot  x  \cdot  -1  \cdot  -1  +  10  \cdot  \cos  10  \cdot  x  \cdot  0 )= 0
\end{dmath}
\begin{dmath}
\text{Этому вас должны были научить в садике: } \frac {dx}{d}( \cos  x  \cdot  -1 )= 0
\end{dmath}
\begin{dmath}
\text{Нет никаких проблем в том, что: } \frac {dx}{d}( -1 )= 0
\end{dmath}
\begin{dmath}
\text{По всем математическим утверждениям: } \frac {dx}{d}( \cos  x )= 0
\end{dmath}
\begin{dmath}
\text{: } \frac {dx}{d}( x )= 0
\end{dmath}
\begin{dmath}
\text{По всем математическим утверждениям: } \frac {dx}{d}( 1  \cdot  \sin  x  \cdot  -1 )= 0
\end{dmath}
\begin{dmath}
\text{Просто как швейцарское полено: } \frac {dx}{d}( 1  \cdot  \sin  x  \cdot  -1  \cdot  -1  +  \cos  x  \cdot  0 )= 0
\end{dmath}
\begin{dmath}
\text{Просто как швейцарское полено: } \frac {dx}{d}( 1  \cdot  \sin  x  \cdot  -1  \cdot  -1  +  \cos  x  \cdot  0  +  0  \cdot  10  \cdot  10  \cdot  10  \cdot  \cos  10  \cdot  x  \cdot  -1  +  0.1  \cdot  0  \cdot  10  \cdot  10  \cdot  \cos  10  \cdot  x  \cdot  -1  +  10  \cdot  0  \cdot  10  \cdot  \cos  10  \cdot  x  \cdot  -1  +  10  \cdot  0  \cdot  \cos  10  \cdot  x  +  10  \cdot  0  \cdot  x  +  10  \cdot  1  \cdot  \sin  10  \cdot  x  \cdot  -1  \cdot  -1  +  10  \cdot  \cos  10  \cdot  x  \cdot  0 )= 0
\end{dmath}

\begin{equation}
f(x)= 2  +  -50.5  \cdot  x  ^ { 2 }+o(x^3)
\end{equation}
\begin{dmath}
\text{Сам ряд Тейлора: } \frac {dx}{d}( 2  +  -50.5  \cdot  x  ^ { 2 })= 0
\end{dmath}
\begin{dmath}
\text{: } \frac {dx}{d}( 2  +  -50.5  \cdot  x  ^ { 2 })= 0
\end{dmath}
\begin{dmath}
\text{Нет никаких проблем в том, что: } \frac {dx}{d}( -50.5  \cdot  x  ^ { 2 })= 0
\end{dmath}
\begin{dmath}
\text{Просто как швейцарское полено: } \frac {dx}{d}( x  ^ { 2 })= 0
\end{dmath}
\begin{dmath}
\text{По всем математическим утверждениям: } \frac {dx}{d}( x )= 0
\end{dmath}
\begin{dmath}
\text{По всем математическим утверждениям: } \frac {dx}{d}( 1  \cdot  2  \cdot  x  ^ { 2  -  1 })= 0
\end{dmath}
\begin{dmath}
\text{По всем математическим утверждениям: } \frac {dx}{d}( -50.5 )= 0
\end{dmath}
\begin{dmath}
\text{Просто как швейцарское полено: } \frac {dx}{d}( 0  \cdot  x  ^ { 2 } +  -50.5  \cdot  1  \cdot  2  \cdot  x  ^ { 2  -  1 })= 0
\end{dmath}
\begin{dmath}
\text{И тд. и тп.: } \frac {dx}{d}( 2 )= 0
\end{dmath}
\begin{dmath}
\text{: } \frac {dx}{d}( 0  +  0  \cdot  x  ^ { 2 } +  -50.5  \cdot  1  \cdot  2  \cdot  x  ^ { 2  -  1 })= 0
\end{dmath}
\includegraphics{1}

\end{document}