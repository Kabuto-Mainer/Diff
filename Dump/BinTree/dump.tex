\documentclass[a4paper,12pt]{article}

\usepackage[utf8]{inputenc}
\usepackage[russian]{babel}
\usepackage{amsmath}
\usepackage{breqn}
\usepackage{graphicx}
\graphicspath{ {Dump/BinTree/Graphic/} }
\usepackage{amssymb}
\usepackage{caption}
\begin{document}
\begin{dmath}
\text{Просто как швейцарское полено: } \frac {dx}{d}( \sin  x )= 0
\end{dmath}
\begin{dmath}
\text{Перед созданием ряда Тейлора: } \frac {dx}{d}( \sin  x )= 0
\end{dmath}
\begin{dmath}
\text{Нулевая производная для построения Тейлора: } \frac {dx}{d}( \sin  x )= 0
\end{dmath}
\begin{dmath}
\text{По всем математическим утверждениям: } \frac {dx}{d}( \sin  x )= 0
\end{dmath}
\begin{dmath}
\text{Просто как швейцарское полено: } \frac {dx}{d}( x )= 0
\end{dmath}
\begin{dmath}
\text{Ёхоу, и бутылка: } \frac {dx}{d}( 1  \cdot  \cos  x )= 0
\end{dmath}
\begin{dmath}
\text{1 производная для построения Тейлора: } \frac {dx}{d}( 1  \cdot  \cos  x )= 0
\end{dmath}
\begin{dmath}
\text{И тд. и тп.: } \frac {dx}{d}( \cos  x )= 0
\end{dmath}
\begin{dmath}
\text{Legendary: } \frac {dx}{d}( x )= 0
\end{dmath}
\begin{dmath}
\text{Realy hasd integral: } \frac {dx}{d}( 1  \cdot  \sin  x  \cdot  (-1) )= 0
\end{dmath}
\begin{dmath}
\text{2 производная для построения Тейлора: } \frac {dx}{d}( 1  \cdot  \sin  x  \cdot  (-1) )= 0
\end{dmath}
\begin{dmath}
\text{Вжух, и готово: } \frac {dx}{d}( \sin  x  \cdot  (-1) )= 0
\end{dmath}
\begin{dmath}
\text{Пара пара пара пам: } \frac {dx}{d}( (-1) )= 0
\end{dmath}
\begin{dmath}
\text{Ёхоу, и бутылка: } \frac {dx}{d}( \sin  x )= 0
\end{dmath}
\begin{dmath}
\text{Realy hasd integral: } \frac {dx}{d}( x )= 0
\end{dmath}
\begin{dmath}
\text{Нет никаких проблем в том, что: } \frac {dx}{d}( 1  \cdot  \cos  x )= 0
\end{dmath}
\begin{dmath}
\text{Нет никаких проблем в том, что: } \frac {dx}{d}( 1  \cdot  \cos  x  \cdot  (-1)  +  \sin  x  \cdot  0 )= 0
\end{dmath}
\begin{dmath}
\text{3 производная для построения Тейлора: } \frac {dx}{d}( 1  \cdot  \cos  x  \cdot  (-1)  +  \sin  x  \cdot  0 )= 0
\end{dmath}
\begin{dmath}
\text{Пара пара пара пам: } \frac {dx}{d}( \cos  x  \cdot  (-1) )= 0
\end{dmath}
\begin{dmath}
\text{Ёхоу, и бутылка: } \frac {dx}{d}( (-1) )= 0
\end{dmath}
\begin{dmath}
\text{И тд. и тп.: } \frac {dx}{d}( \cos  x )= 0
\end{dmath}
\begin{dmath}
\text{Ту ту, ту ту, ту, ту ту ту, ту ту: } \frac {dx}{d}( x )= 0
\end{dmath}
\begin{dmath}
\text{За: } \frac {dx}{d}( 1  \cdot  \sin  x  \cdot  (-1) )= 0
\end{dmath}
\begin{dmath}
\text{: } \frac {dx}{d}( 1  \cdot  \sin  x  \cdot  (-1)  \cdot  (-1)  +  \cos  x  \cdot  0 )= 0
\end{dmath}
\begin{dmath}
\text{4 производная для построения Тейлора: } \frac {dx}{d}( 1  \cdot  \sin  x  \cdot  (-1)  \cdot  (-1)  +  \cos  x  \cdot  0 )= 0
\end{dmath}
\begin{dmath}
\text{Нет никаких проблем в том, что: } \frac {dx}{d}( \sin  x  \cdot  (-1)  \cdot  (-1) )= 0
\end{dmath}
\begin{dmath}
\text{Ту ту, ту ту, ту, ту ту ту, ту ту: } \frac {dx}{d}( (-1) )= 0
\end{dmath}
\begin{dmath}
\text{Ёхоу, и бутылка: } \frac {dx}{d}( \sin  x  \cdot  (-1) )= 0
\end{dmath}
\begin{dmath}
\text{Просто как швейцарское полено: } \frac {dx}{d}( (-1) )= 0
\end{dmath}
\begin{dmath}
\text{: } \frac {dx}{d}( \sin  x )= 0
\end{dmath}
\begin{dmath}
\text{Пара пара пара пам: } \frac {dx}{d}( x )= 0
\end{dmath}
\begin{dmath}
\text{Dead node: } \frac {dx}{d}( 1  \cdot  \cos  x )= 0
\end{dmath}
\begin{dmath}
\text{Вжух, и готово: } \frac {dx}{d}( 1  \cdot  \cos  x  \cdot  (-1)  +  \sin  x  \cdot  0 )= 0
\end{dmath}
\begin{dmath}
\text{Ту ту, ту ту, ту, ту ту ту, ту ту: } \frac {dx}{d}( 1  \cdot  \cos  x  \cdot  (-1)  +  \sin  x  \cdot  0  \cdot  (-1)  +  \sin  x  \cdot  (-1)  \cdot  0 )= 0
\end{dmath}
\begin{dmath}
\text{5 производная для построения Тейлора: } \frac {dx}{d}( 1  \cdot  \cos  x  \cdot  (-1)  +  \sin  x  \cdot  0  \cdot  (-1)  +  \sin  x  \cdot  (-1)  \cdot  0 )= 0
\end{dmath}

\begin{equation}
f{0}(x)= 0 +o(x^0)
\end{equation}

\begin{equation}
f{1}(x)= x +o(x^1)
\end{equation}

\begin{equation}
f{2}(x)= x +o(x^2)
\end{equation}

\begin{equation}
f{3}(x)= x  +  (-0.166667)  \cdot  x  ^ { 3 }+o(x^3)
\end{equation}

\begin{equation}
f{4}(x)= x  +  (-0.166667)  \cdot  x  ^ { 3 }+o(x^4)
\end{equation}

\begin{equation}
f{5}(x)= x  +  (-0.166667)  \cdot  x  ^ { 3 } +  0.00833333  \cdot  x  ^ { 5 }+o(x^5)
\end{equation}
\includegraphics{1}
\begin{dmath}
\text{Сам ряд Тейлора: } \frac {dx}{d}( x  +  (-0.166667)  \cdot  x  ^ { 3 } +  0.00833333  \cdot  x  ^ { 5 })= 0
\end{dmath}

\end{document}